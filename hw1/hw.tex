\documentclass{abrice}

\usepackage{bm}
\usepackage{tikz}
\usetikzlibrary{arrows, automata}

\newcommand{\N}{\mathbb{N}}

\author{Anthony Brice}
\title{Comp 454: Homework 1}

\begin{document}
\maketitle

\section{Problem 1.31}
\textit{Claim.} For any language $A$, let $A^R = \{ w^R \mid w \in A \}$ where
$w^R$ is the string $w$ in reverse order. Then if
$A$ is regular, so is $A^R$.

\begin{proof}
  Let $A$ be a regular language. Then there is some DFA $M$ such that $L(M) =
  A$. Then let NFA $M'$ be $M$ with a few changes:
  \begin{itemize}
  \item For each edge in $M$, the direction of the edge is reversed.
  \item The start state of $M$ is the sole accept state of $M'$.
  \item $M'$ has an entirely new start state with $\epsilon$-transitions to the
    states that were the accept states of $M$.
  \end{itemize}
  Then $M'$ recognizes $w^R \in A^R$.
\end{proof}

\section{Problem 1.36}
\textit{Claim.} Let $B_n = \{ a^k \mid k \text{ is a multiple of } n \}$. For
all $n \geq 1$, language $B_n$ is regular.

\begin{proof}
  Let $n \in \N$. Then build a DFA with $n$ states $q_1, q_2, \ldots, q_n$ where
  $q_1$ is the start state and the sole accept state. Let the transition
  function be defined by $\delta(q_i, a) = q_j$ where $j = (i \bmod n) + 1$.
  Then we have constructed a finite automaton accepting $B_n$, thus $B_n$ is
  regular.
\end{proof}

\section{Problem 1.37}
\textit{Claim.} Let $C_n = \{x \mid x \text{ is a binary number that is a
  multiple of } n \}$. Then for each $n \geq 1$, the language $C_n$ is regular.

\begin{proof}
  Let $n \in \N$. Note that for any $x \in C_n$, $n$ divides $x$. Then build a
  DFA with $n$ states $q_1, q_2, \ldots, q_n$ where $q_1$ is the start state and
  the sole accept state. Let the transition function be defined by $\delta(q_i,
  a \in \{0,1\}) = q_j$ where $j = ((2i + a) \bmod n) + 1$. Then we have
  constructed a finite automaton accepting $C_n$, thus $C_n$ is regular.
\end{proof}

\section{Problem 1.40}

\subsection{Part a)}

\emph{Claim.} The class of regular languages is closed under the
$\text{NOPREFIX}$ operation defined by
\[
  \text{NOPREFIX}(A) = \{ w \in A \mid \text{no proper prefix of } w \text{ is a
  member of } A\}\, .
\]

\begin{proof}
  Let $A$ be a regular language and $M = (Q, \Sigma, \delta, q_0, F)$ be a DFA
  recognizing $A$. Then $M' = (Q, \Sigma, \delta', q_0, F)$ where
  \[
    \delta'(r \in Q, a \in \Sigma) =
    \begin{cases}
      \{\delta(r,a)\} & \text{if } r \notin F \\
      \O & \text{if } r \in F
    \end{cases}
  \]
  is a finite automaton recognizing $\text{NOPREFIX}(A)$.
\end{proof}

\subsection{Part b)}

\emph{Claim.} The class of regular languages is closed under the
$\text{NOEXTEND}$ operation defined by
\[
  \text{NOEXTEND}(A) = \{ w \in A \mid w \text{ is not the proper prefix of any
    string in } A\}\, .
\]

\begin{proof}
  Let $A$ be a regular language and $M = (Q, \Sigma, \delta, q_0, F)$ be a DFA
  recognizing $A$. Then construct $M' = (Q, \Sigma, \delta, q_0, F')$ where $F'$
  is the set of all $q \in F$ for which there exists no path from $q$ to an
  accept state. Then $M'$ is a finite automaton recognizing $NOEXTEND(A)$.
\end{proof}

\section{Problem 1.41}
\textit{Claim.} The class of regular languages is closed under the \textit{perfect
shuffle} defined by
\[ \{ w \mid w = a_1b_1 \cdots a_k b_k, \text{ where } a_1
\cdots a_k \in A, b_1 \cdots b_k \in B, \text{ each } a_i, b_i \in \Sigma \} \]
for languages $A$ and $B$.

\begin{proof}
  Let $A$ and $B$ be regular languages, $M_A = (Q_A, \Sigma_A, \delta_A, q_A,
  F_A)$ be a DFA recognizing $A$, and $M_B = (Q_B, \Sigma_B, \delta_B, q_B,
  F_B)$ be a DFA recognizing $B$. Then construct $M_{A \times B} = (Q, \Sigma, \delta,
  q_0, F)$ defined by
  \begin{itemize}
  \item $Q = Q_A \times Q_B \times \{A,B\}$,
  \item $\Sigma = \Sigma_A \cup \Sigma_B$,
  \item $q_0 = (q_A, q_B, A)$,
  \item $F = F_A \times F_B \times \{A\}$,
  \item
    \[
      \delta((x \in Q_A, y \in Q_B, z \in \{A, B\}), a \in \Sigma) =
      \begin{cases}
        (\delta_A(x,a),y,B) & \text{if } z = A \\
        (x, \delta_B(y,a), A) & \text{if } z = B\, .
      \end{cases}
    \]
  \end{itemize}
  Then $M_{A \times B}$ is a finite automaton recognizing the \emph{perfect
    shuffle} of $A$ and $B$.
\end{proof}

\section{Problem 1.42}

\emph{Claim.} The class of regular languages is closed under the \emph{shuffle}
defined by
\[ \{ w \mid w = a_1b_1 \cdots a_k b_k, \text{ where } a_1
\cdots a_k \in A, b_1 \cdots b_k \in B, \text{ each } a_i, b_i \in \Sigma^*
\} \]
for languages $A$ and $B$.

\begin{proof}
  Let $A$ and $B$ be regular languages, $M_A = (Q_A, \Sigma_A, \delta_A, q_A,
  F_A)$ be a DFA recognizing $A$, and $M_B = (Q_B, \Sigma_B, \delta_B, q_B,
  F_B)$ be a DFA recognizing $B$. Then construct $N_{A \times B} = (Q, \Sigma,
  \delta, q, F)$ defined by
  \begin{itemize}
  \item $Q = (Q_A \times Q_B) \cup \{q_0 \}$,
  \item $\Sigma = \Sigma_A \cup \Sigma_B$,
  \item $q = q_0$,
  \item $F = (F_A \times F_B) \cup \{ q_0\}$,
  \item
    \[
      \delta(z,a) =
      \begin{cases}
        (q_A, q_B) & \text{if } z = q_0 \text{ and } a = \epsilon \\
        (\delta_A(x,a), y) & \text{if } z = (x,y) \in Q_A \times Q_B \text{ and
        } a \in \Sigma_A \\
        (x, \delta_B(y,a)) & \text{if } z = (x,y) \in Q_A \times Q_B \text{ and
        } a \in \Sigma_B\, .
      \end{cases}
    \]
  \end{itemize}
  Then $N_{A \times B}$ is a finite automaton recognizing the \emph{shuffle} of
  $A$ and $B$
\end{proof}
\end{document}

  % Note that it is sufficient to prove this in the $n = 1$ case since any $B_{n >
  %   1}$ is necessarily a subset of $B_1$ and is thus regular if $B_1$ is
  % regular.

  % Consider the DFA in Fig.~\ref{fig:136}. Clearly the DFA recognizes any string
  % in $\{ a^n \mid n \geq 1\}$, so for all $n \geq 1$, $B_n$ is regular.
  % \begin{figure}
  %   \centering
  %   \label{fig:136}
  %   \caption{A DFA recognizing $B_n$.}
  %   \begin{tikzpicture}[->, >=stealth', shorten >=1pt, auto, node
  %     distance=8em, semithick, initial text={}]
  %     \node[initial, state, accepting] (q1) {$q_1$};
  %     %\node[accepting, state] (q2) [right of=q1] {$q_2$};

  %     %\path (q1) edge node {$a$} (q2);
  %     \path (q1) edge [loop above] node {$a$} (q1);
  %   \end{tikzpicture}
  % \end{figure}

  % We formally define the machine in Fig.~\ref{fig:136} below.

  % \begin{align*}
  %   & Q = \{ q_1 \} \\
  %   & \Sigma = \{ a \} \\
  %   & \delta(q_1, a) = q_1 \\
  %   & q_1 \text{ is the } \text{start state} \\
  %   & F = \{ q_1 \}\, .
  % \end{align*}
