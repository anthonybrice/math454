\documentclass{abrice}

\title{Comp 454: Homework 2}
\author{Anthony Brice}

\usepackage{tikz}
\usetikzlibrary{arrows, automata}

\newcommand{\comp}[1]{\overline{#1}}%

\begin{document}
\maketitle

\section{Problem 1.39}

\emph{Claim.} For every $k > 1$, a language $A_k \subseteq {\{0,1\}}^*$ exists
that is recognized by a DFA with $k$ states but not by any with only $k - 1$
states.

\begin{proof}
  Let $M_k = (\{q_0, q_1, \ldots, q_{k-1}\}, \{0,1\}, \delta_k, q_0,
  \{q_{k-2}\})$ where
  \[
    \delta_k(q_i,a) =
    \begin{cases}
      q_i & \text{if } a = 1 \\
      q_{i + 1} & \text{if } a = 0 \text{ and } i \neq k - 1
      \\
      q_{k - 2} & \text{if } a = 0 \text{ and } i  = k -1\, .
    \end{cases}
  \]
  Then let $A_k = L(M_k)$. If $k$ is even, $A_k = \{ w \mid w \in {\{0,1\}}^*, w
  $ has an even number of $0$'s greater than $k - 3\}$. If $k$ is odd, $A_k = \{
  w \mid w \in {\{0,1\}}^*, w$ has an odd number of $0$'s greater than $k -
  3\}$. Then $A_k$ is recognized by the DFA $M_k$ with $k$ states.

  Let $M_{k - 1}$ be any DFA with $k - 1$ states. If $k$ is even, $M_{k - 1}$
  cannot possibly recognize $A_k$ because to accept some string with an even
  number of $0$'s means $M_{k-1}$ must also accept some string with an odd number
  of $0$'s. Similarly if $k$ is odd, $M_{k-1}$ cannot possibly recognize $A_k$
  because to accept some string with an odd number of $0$'s means $M_{k-1}$
  must also accept some string with an even number of $0$'s.

  Thus $A_k$ is recognized by a DFA with $k$ states but not by any with $k - 1$
  states.
\end{proof}

\section{Problem 1.46}

\subsection{Part a)}

\emph{Claim.} $B = \{0^n 1^m 0^n \mid m,n \geq 0\}$ is not regular.

\begin{proof}
  Assume $B$ is regular. Then let $p$ be the pumping length of $B$ given by the
  pumping lemma. Choose $s$ to be the string $0^p 1 0^p$. Then there exists $xyz
  = s$ where for any $i \geq 0$, $x y^i z \in B$. Since we have $p$ $0$'s on the
  left of $s$, $\abs{xy} \leq p$, and $\abs{y} > 0$, we have that $xy$ must be
  of the form $0^+$, which further gives us that $y$ must be of the form $0^+$.
  Then clearly $xyyz \notin B$ since $xyyz$ is not of the form $0^n 1^m 0^n \mid
  m,n \geq 0$. Thus we have found a contradiction, so $B$ is not regular.
\end{proof}

\subsection{Part b)}

\emph{Claim.} $B = \{0^m 1^n \mid m \neq n\}$ is not regular.

\begin{proof}
  Assume $B$ is regular. Then $\comp{B} \cap \{0^* 1^*\} = \{0^k 1^k \mid k \geq
  0\}$ must be regular. But Example 1.73 proves that $\{0^k 1^k \mid k \geq 0\}$
  is not regular, so we have found a contradiction. Thus $B$ is not regular.
\end{proof}

\subsection{Part c)}

\emph{Claim.} $B = \{ w \mid w \in {\{0,1\}}^*$ is not a palindrome$\}$ is not
regular.

\begin{proof}
  Assume $B$ is regular. Then $\comp{B} \cap \{0^* 1^* 0^*\} = \{ 0^n 1^m 0^n
  \mid n,m \geq 0\}$ must be regular. But we have already proved that $\{0^n 1^m
  0^n\}$ is not regular, so we have found a contradiction. Thus $B$ is not
  regular.
\end{proof}

\subsection{Part d)}

\emph{Claim.} $B = \{ wtw \mid w,t \in {\{0,1\}}^+ \}$ is not regular.

\begin{proof}
  Assume $B$ is regular. Then let $p$ be the pumping length of $B$ given by the
  pumping lemma. Choose $s = 0^p 1 1 0^p 1$. Then there exists $xyz = s$
  where for any $i \geq 0$, $x y^i z \in B$. Since we have $p$ $0$'s on the left
  of $s$ and since $\abs{xy} \leq p$, $xy$ must be of the form $0^+$, and since
  $\abs{y} > 0$, we further have that $y$ must be of the form $0^+$. Then
  clearly $xyyz \notin B$ since $xyyz$ cannot be of the form $wtw \mid w,t \in
  {\{0,1\}}^+$. Thus we have found a contradiction, so $B$ is not regular.
\end{proof}

\section{Problem 1.51}

\emph{Claim.} Let $L$ be any language. Then $\equiv_L$ is an equivalence
relation.

\begin{proof}
  Let $x$ and $y$ be strings. Then clearly $x \equiv_L x$, so $\equiv_L$ is
  reflexive. Similarly it is obvious from the given rules that $x \equiv_L y
  \Rightarrow y \equiv_L x$, so $\equiv_L$ is symmetric. Then we only have left
  to prove that $\equiv_L$ is transitive.

  Let $z$ also be a string, and assume $x \equiv_L y$ and $y \equiv_L z$.
  Then for every string $s$, $xs \in L \Leftrightarrow ys \in L$, and
  $ys \in L \Leftrightarrow zs \in L$. Then $xs \in L \Leftrightarrow zs \in L$,
  so $x \equiv_L z$. Thus $\equiv_L$ is transitive.

  Since $\equiv_L$ is reflexive, symmetric, and transitive, we have that
  $\equiv_L$ is an equivalence relation.
\end{proof}

\end{document}
