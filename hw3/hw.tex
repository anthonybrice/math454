\documentclass{abrice}

\title{Comp 454: Homework 3}
\author{Anthony Brice}

\begin{document}
\maketitle

\section{Exercise 2.4}

\begin{enumerate}[label=\textbf{\alph*)}]
\item
  \begin{align*}
    S &\rightarrow R1R1R1 \\
    R &\rightarrow 0R \mid 1R \mid \epsilon\, .
  \end{align*}
\item
  \begin{align*}
    S &\rightarrow 0R0 \mid 1R1 \\
    R &\rightarrow 0R \mid 1R \mid \epsilon \, .
  \end{align*}
\item
  \begin{align*}
    S &\rightarrow 1R \mid 0R \\
    R &\rightarrow 00R \mid 01R \mid 10R \mid 11R \mid \epsilon\, .
  \end{align*}
\item
  \begin{align*}
    S &\rightarrow 0 \mid 0S0 \mid 0S1 \mid 1S0 \mid 1S1\, .
  \end{align*}
\item
  \begin{align*}
    S &\rightarrow 0 \mid 1 \mid \epsilon \mid 0S0 \mid 1S1\, .
  \end{align*}
\item
  \begin{align*}
    S \rightarrow S\, .
  \end{align*} \marginnote{How would one prove this generates the empty set?}
\end{enumerate}

\section{Exercise 2.6}
\begin{enumerate}[label=\textbf{\alph*)}]
\item
  \begin{align*}
    S &\rightarrow RaR \\
    R &\rightarrow RR \mid aRb \mid bRa \mid a \mid \epsilon\, .
  \end{align*}
\item
  \begin{align*}
    S &\rightarrow R \mid T \\
    R &\rightarrow aRb \mid Q \mid U \\
    T &\rightarrow PbaP \\
    Q &\rightarrow aQ \mid a \\
    U &\rightarrow Ub \mid b \\
    P &\rightarrow PP \mid a \mid b \mid \epsilon
  \end{align*}
\item
  \begin{align*}
    S &\rightarrow TX \\
    T &\rightarrow 0T0 \mid 1T1 \mid \#X \\
    X &\rightarrow 0X \mid 1X \mid \epsilon\, .
  \end{align*}
\item
  \begin{align*}
    S &\rightarrow R \# T \# R \mid T \# R \mid R \# T \mid T \\
    T &\rightarrow aTa \mid bRb \mid \# \mid \# R \# \\
    R &\rightarrow a R \mid b R \mid \# R \mid \epsilon
  \end{align*}
\end{enumerate}

\section{Exercise 2.9}

The following grammar generates the language $A = \{a^i b^j c^k
\mid i = j \text{ or } j = k \text{ where } i,j,k \geq 0 \}$:
\begin{align*}
  S &\rightarrow R C \mid A T \\
  R &\rightarrow a R b \mid \epsilon \\
  C &\rightarrow c C \mid \epsilon \\
  T &\rightarrow b T c \mid \epsilon \\
  A &\rightarrow a A \mid \epsilon\, .
\end{align*}

\noindent
\emph{Claim.} This grammar is ambiguous.

\begin{proof}
  Consider the empty string $\epsilon$. Clearly $\epsilon \in A$, and $\epsilon$
  can be generated by following either substitution for $S$, thus the grammar is
  ambiguous.
\end{proof}

\section{Problem 2.20}

\emph{Claim.} Let $A/B = \{w \mid wx \in A \text{ for some } x \in B \}$ where
$A$ is context free and $B$ is regular. Then $A/B$ must be context free.

\begin{proof}
  Let $W$ be a PDA recognizing $A$, and $X$ be a DFA recognizing $B$. Let $wx
  \in A$ for some $x \in B$. Note that we could construct a potentially new PDA
  $W'$ recognizing $A$ such that $W'$ starts parsing any $wx$ as $W$ would, and
  when it reaches the start of a string recognized by $X$, it finishes parsing
  the rest exactly as $X$ would. Then we can construct a new PDA $W'/X$ from
  $W'$ by stripping away the states associated with parsing a string in $X$ and
  making the states just before which $W'$ would transition to parsing a string $X$ be
  the new accept states of $W'/X$. Then $W'/X$ is PDA recognizing $w \in A/B$,
  so $A/B$ is context free.
\end{proof}

\end{document}
